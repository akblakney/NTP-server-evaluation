\documentclass[letterpaper, 11pt]{article}
\usepackage{latexsym}
\usepackage{amssymb}
\usepackage{times}
\usepackage{amsmath,amsfonts,amsthm, pifont}
\usepackage{graphicx}
\usepackage{upgreek}
\usepackage{hyperref}  
\usepackage[margin=0.75in]{geometry}
\usepackage{cite}

\begin{document}
     
\title{NTP Server Evaluation: A Mathematical Model}
\author{\textbf{Adam Blakney} \\ 
\\}
\maketitle
        
\section{Estimating a System's Drift Constant}

Time laboratories, such as NIST's Time and Frequency Division, distribute their own realization
of standard time scales such as UTC. One method of the distribution of their time is by the
Network Time Protocol (NTP) method, in which packets are sent over the internet so that
other systems may synchronize their time with that of NIST. However, there is uncertainty
associated with this distribution method, on the order of hundreds of microseconds to
milliseconds, according to NIST. As such, we model the output of NIST's time distribution
via NTP as a non-deterministic process with some uncertainty.

Let $N(t)$ denote the offset between a realization of UTC laboratory time---for example
UTC(NIST)---and the distributed time via NTP, at some time $t$.
That is, $N(t)  = t
- \text{NTP}(t)$, where $t$ is NIST's realization of UTC time and $\text{NTP}(t)$ is the
time distributed by NIST's NTP server at time $t$. Thus, $N(t)$ gives the error associated
with the distribution of the time via NTP at time $t$.
Because the $N(t) $ are non-deterministic, we will model them as random variables. In particular,
we assume that the $N(t)$ are distributed i.i.d. according to a Normal distribution with mean
zero and variance $\sigma^2$, for all $t$. 

Now, let $S(t)$ be the system time at time $t$. By "time $t$," we mean the time as determined
by a time laboratory such as NIST. Because computer clocks are very inaccurate
and unstable compared to laboratory clocks, they will exhibit the phenomenon of "clock drift,"
whereby they drift apart from a standardized timescale such as UTC and its realizations
by laboratory clocks. We will assume that the system clock exhibits a linear frequency drift,
i.e. $S(t) = Dt + b$ for a drift constant $D$ and constant $b$.

When we query a NTP server, we are given $\text{NTP}(t) = t - N(t)$ in response. We can also
easily calculate the offset between the NTP time and the system clock's time, with
$\text{NTP}(t) - S(t) = t - N(t) - S(t)$. For a time $t_i$, let us
denote this offset quantity by $\Delta(t_i)
\equiv \text{NTP}(t_i)-S(t_i) = t_i - N(t_i) - S(t_i) $, and note that $\Delta(t_i)$ is an
observable quanity,
like $\text{NTP}(t)$ and $S(t)$. Also note that $\Delta(t_i)$ is distributed Normally with
mean $t_i - S(t_i)  = (1-D)t_i - b$ and variance $\sigma^2$.

Now, we wish to estimate the constants $D$ and $b$. In order to obtain estimates for these
constants, we will have to observe many instances of $\Delta(t)$ at different times. We will
then use the maximum likelihood method to estimate $D$ and $b$.

Consider $n$ observations of our observed quantity, $\Delta(t_1), ..., \Delta(t_n)$.
Since the $\Delta(t_i)$ are distributed Normally with mean $(1-D)t_i - b$
and variance $\sigma^2$,
The likelihood function for our data is.
\begin{align}
    \mathcal{L}(D, b, \sigma^2; \Delta(t_1),..., \Delta(t_n)) &= \prod_{i=1}^n
    (2\pi \sigma^2)^{-1/2} \text{exp}[- \frac{1}{2\sigma^2} (\Delta(t_i)
    - (1-D)t_i +b)^2] \\
    &=  (2\pi \sigma^2)^{-n/2} \text{exp}[- \frac{1}{2\sigma^2} \sum_{i=1}^n 
    (\Delta(t_i) - (1-D)t_i + b)^2 ]
\end{align}
And note that the right-hand expression is maximized when
$\sum_{i=1}^n 
(\Delta(t_i) - (1-D)t_i + b)^2$
is minimized. Further note that this is a sum of squares of differences between
$\Delta(t_i)$ and $(1-D)t_i - b$. Therefore, if we plot pairs $(t_i, \Delta(t_i))$ and
perform a linear least squares fit on this data, we will obtain a line with slope $1-D$ and 
with an intercept of $-b$, allowing us to estimate both of these quantities. It might be noted that $t_i$ is not an observable quantity, which
is indeed true. However, recall that $t = N(t) + \text{NTP}(t)$, and that $N(t)$ has mean zero.
Therefore, $\text{NTP}(t)$ is an unbiased estimator for $t$ since
$\mathbb{E}(\text{NTP}(t)) = \mathbb{E}(t - N(t)) = t$. Thus, we may instead plot observable
pairs $(\text{NTP}(t_i), \Delta(t_i))$ and perform our linear least squares fit procedure on this
data.

To summarize, we have shown that estimates of the drift constant $D$ and the constant $b$
can be obtained by plotting observable pairs $(\text{NTP}(t_i), \Delta(t_i))$ and then
performing a linear least squares fit on the data. This will yield a line with slope $1-D$ and
intercept $-b$, allowing us to estimate $D$ and $b$ with $\hat{D} = 1 - \text{slope}$ and
$\hat{b} = - \text{intercept}$.



\section{Evaluating NTP Servers}
Once an estimate of the constants $D$ and $b$, we obtain the true time $t$ from our system
time by correcting for the system clock's drift, with $t = 2$




\pagebreak

\section{References}

\bibliography{mybib}
\bibliographystyle{plain}

\end{document}